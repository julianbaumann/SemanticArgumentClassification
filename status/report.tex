\documentclass[runningheads]{llncs}

\usepackage[margin=1.5in]{geometry}
\usepackage{ngerman}
\usepackage{gantt}
\usepackage{listings}
\usepackage{cite}

\definecolor{airforce}{rgb}{0.36, 0.54, 0.66}
\definecolor{amber}{rgb}{1.0, 0.75, 0.0}
\definecolor{pastelgreen}{rgb}{0.01, 0.75, 0.24}
\definecolor{deepcarrot}{rgb}{0.91, 0.41, 0.17}

\begin{document}

\lstset{language=Python}

\title{Semantic Argument Classification}

\author{Julian Baumann, Kevin Decker, Maximilian M\"uller-Eberstein}

\institute{Ruprecht-Karls-Universit\"at Heidelberg}

\maketitle

\section{Einleitung}
\textit{Wem} ist durch \textit{wen}, \textit{was}, \textit{wo}, \textit{wie}, \textit{wann} widerfahren? Genau diese Fragen beantwortet die semantische Argumentklassifikation. Den Argumenten des Verbs ihre semantischen Rollen zuzuweisen ist f\"ur das tiefere Verst\"andnis textueller Daten unumg\"anglich. Um diese Aufgabe bestm\"oglich zu l\"osen, testen wir verschiedene Klassifikationsalgorithmen und orientieren uns dabei an Support Vector Learning for Semantic Argument Classification\cite{svm}.
\subsection{Organisatorisches}
\begin{gantt}{13}{10}
    \begin{ganttitle}
      \titleelement{Oktober}{1}
      \titleelement{November}{2}
      \titleelement{Dezember}{3}
      \titleelement{Januar}{2}
      \titleelement{Februar}{2}
    \end{ganttitle}
    \begin{ganttitle}
      \titleelement{22.10.}{1}
      \titleelement{05.11.}{1}
      \titleelement{19.11.}{1}
      \titleelement{3.12.}{1}
      \titleelement{17.12.}{1}
      \titleelement{31.12.}{1}
      \titleelement{14.1.}{1}
      \titleelement{28.1.}{1}
      \titleelement{11.2.}{1}
      \titleelement{28.2.}{1}
    \end{ganttitle}
    \ganttbar[color=pastelgreen]{Team \& Thema}{0}{1}
    \ganttbar[color=pastelgreen]{Paper \& Ressourcen}{1}{1}
    \ganttbar[color=pastelgreen]{Umsetzungsplanung}{2}{2}
    \ganttmilestone[color=airforce]{Statusbericht}{4}
    \ganttbar[color=deepcarrot]{Feature-Extraktion}{4}{3}
    \ganttbar[color=deepcarrot]{Experimente (Weka)}{6}{2}
    \ganttmilestone[color=airforce]{Grundarchitektur +1 Feature}{6}
    \ganttbar[color=pastelgreen]{Pr\"asentation vorbereiten}{6}{1}
    \ganttbar[color=pastelgreen]{Endbericht und Dokumentation}{6}{2}
    \ganttmilestone[color=airforce]{Endbericht}{9}
    \ganttbar[color=amber]{Puffer}{8}{1}
\end{gantt}

\section{Grundlagen}
\textbf{TODO:} Kevin
\subsection{Daten}
PropBank in NLTK (Gr\"o\ss{}e)\\
112917 PropBank-Instanzen/S\"atze\\
292975 Argumente
\subsection{Tools}
Python 3.4\\
NLTK 3.0.0\\
Weka 3.7.11
\subsection{Algorithmen}
Classifiers\\
SVM\\
NaiveBayes\\
J48
\subsection{Vergleichsgrundlagen}
Paper: Support Vector Learning for Semantic Argument Classification\\
PropBanks goldene Annotierung\\
Trainings- und Testdaten aufsplitten ~(60\% Training, 20\% Test, 20\% Development) 

\section{Hauptteil}
\subsection{Zielsetzung}
\textbf{TODO:} Julian\\
die automatische Zuordnung mittels Supervised Learning von\\
\\
\begin{table}
\centering
\begin{tabular}{|c|l|}
\hline 
ARG0 & agent \\ 
\hline 
ARG1 & patient \\ 
\hline 
ARG2 & instrument, benefactive, attribute \\ 
\hline 
ARG3 & starting point, benefactive, attribute \\ 
\hline 
ARG4 & ending point \\ 
\hline 
ARGM & modifier \\ 
\hline 
\end{tabular}
\end{table}
 
\subsection{Umsetzung}
\textbf{TODO:} Julian\\
Feature-Extraktion mit Python:
\begin{lstlisting}[frame=lines]
featureList = [...] # zu extrahierende Features
for pbInstance in pbInstances :
   for pbArg in pbInstance.arguments :
      features = []
      for feature in featureList :
         featureList.append(extFeature(feature, pbArg, pbInstance))
# write features to file in ARFF
\end{lstlisting}

\subsubsection{Features}
\begin{itemize}
\item predicate : nominal
\item path : nominal
\item phrase type : nominal
\item position(before/after) : boolean
\item voice(active/passive) : boolean
\item headword : nominal
\item subcategorization : nominal
\end{itemize}
siehe Support Vector Learning for Semantic Argument Classification

\subsection{Evaluation}
\textbf{TODO:} Max\\
Evaluation gegen Support Vector Learning for Semantic Argument Classification

\section{Ausblick}
\textbf{TODO:} Max

\section{Literatur}
\bibliography{lit}{}
\bibliographystyle{plain}

\end{document}
